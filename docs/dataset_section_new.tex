\section{Datasets}\label{app:bench-eq}
All datasets and split definitions used in this work are publicly available in the
\href{https://github.com/yoelt11/eff-physics-learn-dataset}{Efficient Physics Learning Dataset repository}.
The repository provides detailed download instructions, dataset descriptions, and
programmatic usage examples. We summarize the two modalities used throughout this work below.

\subsection{Dataset Modalities}

\paragraph{Standard}
In the standard modality, samples are split randomly into train and test sets without using
parameter-based separation. We follow a fixed budget of $n_{\text{train}}=25$ training samples and
$n_{\text{test}}=40$ test samples per equation with seed $0$, providing a consistent baseline
across datasets.

\paragraph{Parametric}
In the parametric modality, each dataset is split by parameter values rather than by individual
space--time samples. A few-shot training set is formed with $n_{\text{train}}=10$ parameter
configurations, and the remaining samples are divided into interpolation and extrapolation sets
based on their distance in solution space. We compute a PCA embedding of the solution fields
(5 components) and measure the nearest-neighbor distance from each candidate sample to the training
set. Samples are then split at the median distance: those below the threshold form the interpolation
set (closer to training solutions), while those above form the extrapolation set (farther from
training solutions). This solution-space percentile method directly measures generalization difficulty
and provides robust separation across all parameter dimensionalities, unlike parameter-space geometric
approaches that can fail for nonlinear PDEs. For each equation, we target $n_{\text{interp}}=20$ and
$n_{\text{extrap}}=20$ samples per split, with balanced sampling to ensure comparable coverage. We
provide three random seeds ($0,1,2$) to assess variability. This setting isolates how well models
generalize to distant solution regimes (extrapolation) versus nearby solution regimes (interpolation)
from only a few training examples. The full configuration is listed in
Table~\ref{tab:parametric_config}, and the corresponding solution-space distance summary is reported
in Table~\ref{tab:parametric_distances}.

This setup enables a controlled evaluation of the model's ability to interpolate to nearby solutions and extrapolate to distant solutions from only a few training examples.

\begin{table}[h]
\centering
\caption{Parametric split configuration.}
\label{tab:parametric_config}
\begin{tabular}{lll}
\hline
Setting & Value & Notes \\
\hline
$n_{\text{train}}$ & 10 & few-shot training samples \\
$n_{\text{each}}$ & 20 & target samples per split (interp/extrap) \\
balance & true & balanced interp/extrap splits \\
method & solution\_percentile & split by distance in solution space \\
percentile & 50.0 & median distance threshold \\
n\_components & 5 & PCA components for solution embedding \\
seeds & 0, 1, 2 & three splits for reproducibility \\
\hline
\end{tabular}
\end{table}

\begin{table}[h]
\centering
\caption{Mean nearest-neighbor distance to the training set in PCA solution-space (averaged over
seeds). The separation ratio (Extrap/Interp) quantifies the difficulty gap between interpolation
and extrapolation regimes.}
\label{tab:parametric_distances}
\begin{tabular}{lccc}
\hline
Equation & Interp & Extrap & Ratio \\
\hline
Allen--Cahn & 0.49 & 2.72 & 5.59$\times$ \\
Burgers & 5.31 & 12.60 & 2.37$\times$ \\
Convection & 7.90 & 25.58 & 3.24$\times$ \\
Helmholtz 2D & 8.13 & 15.56 & 1.91$\times$ \\
\hline
\end{tabular}
\end{table}